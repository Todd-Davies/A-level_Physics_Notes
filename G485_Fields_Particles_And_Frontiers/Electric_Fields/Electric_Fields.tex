\documentclass{article}

\usepackage[normalem]{ulem}
\usepackage{fancyhdr}
\usepackage[parfill]{parskip}
\usepackage{tikz}
\usepackage{pgfplots}
\usepackage{multicol}
\usepackage[version=3]{mhchem}

\pagestyle{fancyplain}

\pgfplotsset{compat=1.7}

\title{Electric fields}
\author{Todd Davies}
\date{\today}

\begin{document}

\rhead{Electric fields}
\lhead{\today}

\maketitle

\section*{Equations to learn}
\thispagestyle{empty}

%TODO:Write section on equations to learn

\section*{Electric fields}

If an object is charged, then the influence of it's charge spreads into the
space around it and we say that there is an electric field around the object.
Some objects, when placed inside the field will interact with this field. An
example of one such object is gold foil, which will bend towards (be attracted
to) a negatively charged object.

\section*{Electrical field lines}

We can represent an electrical field by drawing it as a series of lines. Their direction shows the direction of the field (from positive to negative) and their spacing indicates the strength of the field (wider spacing for weaker fields). 

\section*{Electric field strength}

The strength of an electric field at a point is defined as:

{\it The electric field strength at a point is the force per unit charge exerted
on a positive charge placed at that point.}

\subsection*{A uniform field}

A uniform field can be set up by putting the terminals of a power supply with a
high potiential difference to two parallel (not touching) metal plates. This
creates an uniform electric field, where the field lines are evenly spaced.

In this situation, the following relationships apply between properties of the
field:

\[
	E \propto V
\]

\[
	E \propto \frac{1}{d}
\]

Which we can combine to get: \marginpar{If you want the magnitude of the
electric field, you can forget the minus sign}

\[
	E = -\frac{V}{d}
\]

Where $d$ is the distance between the plates, $V$ is the potential difference
and $E$ is the strength of the electric field.

%TODO:Do worked example on page 108 in text book

\section*{The force on a charge}

To calculate the force experienced by a charged object in an electric field, we can combine the equation for field strength with the equation for magnitude:

\begin{center}
	$E = \frac{F}{Q} + E = \frac{V}{d} \rightarrow $
	$\frac{F}{Q} = \frac{V}{d}	\rightarrow $
	$F = \frac{VQ}{d}$
\end{center}

If you're getting bogged down in equations, the best thing to do is often to remember that positively charged objects are attracted to negatively charged objects and vice versa.

\section*{Coloumb's law}

\marginpar{Note how similar to the definition of a gravitational field it is...}

{\it Any two point charges exert an electrical force on each other that is
proportional to the product of their charges and inversly proportional to the
square of their seperation.}

%TODO:Finish section on columbs law

\end{document}