\documentclass{article}

\usepackage[normalem]{ulem}
\usepackage{fancyhdr}
\usepackage[parfill]{parskip}
\usepackage{tikz}
\usepackage{pgfplots}
\usepackage{multicol}
\usepackage[version=3]{mhchem}
\pagestyle{fancyplain}

\pgfplotsset{compat=1.7}

\title{Nuclear Physics}
\author{Todd Davies}
\date{\today}

\begin{document}

\rhead{Nuclear Physics}
\lhead{\today}

\maketitle

\section*{Nuclear fission}
\thispagestyle{empty}

Nuclear fission occurs when a large and unstable nucleus splits into two smaller
nuclei and some neutrons. This happens when a neutron hits the large and
unstable nucleus. The neutrons that are released will fission another unstable
nucleus if they hit it, which fuels more fission.

This is what happens in nuclear power plants, and it releases energy.

We can represent fission reactions by equations such as this:
\marginpar{Note the number of protons and neutrons are constant in the reaction}

\[
	^{1}_{0}n + ^{235}_{92}U \rightarrow 
	^{92}_{36}Kr + ^{141}_{56}Ba + 3^{1}_{0}n
\]

There are intermediate reactions in these reactions too since the large unstable
nucleus probably won't decay into two other nuclei straight away, but rather
accept the incoming neutron and then decay:

\[
	^{1}_{0}n + ^{235}_{92}U \rightarrow 
	^{236}_{92}U^{*} \rightarrow
	^{92}_{36}Kr + ^{141}_{56}Ba + 3^{1}_{0}n
\]

\section*{Energy release in fission}

The measured mass of any nucleus is less than the mass of it's consitiuent
particles. This 'loss' of mass is known as the {\it mass defect} of an atom, and
it is linked to the {\it binding energy} of the nucleus.

The binding energy is the minimum energy needed to seperate all of the protons
and neutrons in the nucleus and we can use $E = MC^2$ to work it out.

%TODO:Verify this...

The change in the binding energy between the reactant nucleus of the fission
reaction and the products of the fission reaction results in the release of
energy.

\end{document}