\documentclass{article}

\usepackage[normalem]{ulem}
\usepackage{fancyhdr}
\usepackage[parfill]{parskip}
\usepackage{tikz}
\usepackage{pgfplots}
\usepackage{multicol}
\usepackage[version=3]{mhchem}
\pagestyle{fancyplain}

\pgfplotsset{compat=1.7}

\title{Magnetic fields}
\author{Todd Davies}
\date{\today}

\begin{document}

\rhead{Magnetic fields}
\lhead{\today}

\maketitle

\section*{Equations to learn}
\thispagestyle{empty}

\begin{itemize}
	
	\item The magnetic flux density is defined by:
	
	\[
		B = \frac{F}{IL}
	\]

	\item The magnetic force on a conductor carrying a current is:
	
	\[
		F=BIL
	\]

	\item However, if the direction of propogation of the current isn't
	perpendicular to the line between the North and South poles on the magnet,
	then the equation is:

	\[
		F=BIL\sin{\theta}
	\]

	\item The magnetic force on a moving charged particles is:

	\[
		F=BQv
	\]

	For electrons, the charge $Q$ is of course the elementary charge $e$ giving
	$F=Bev$

    \item We can combine $F=BQv$ with the equation for kinetic energy to create
	an equation to describe an electron travelling in a uniform magnetic field:

	\[
		\frac{mv^2}{r} = Bev
	\]

	\item The velocity of an undeflected charged particle in a region where
	electric and magnetic fields are at right angles is given by:

	\[
		v = \frac{E}{B}
	\]

\end{itemize}

%TODO: Do more work on this topic

\end{document}