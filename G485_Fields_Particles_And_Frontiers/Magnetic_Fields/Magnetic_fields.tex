\documentclass{article}

\usepackage[normalem]{ulem}
\usepackage{fancyhdr}
\usepackage[parfill]{parskip}
\usepackage{tikz}
\usepackage{pgfplots}
\usepackage{multicol}
\usepackage[version=3]{mhchem}
\pagestyle{fancyplain}

\pgfplotsset{compat=1.7}

\title{Magnetic fields}
\author{Todd Davies}
\date{\today}

\begin{document}

\rhead{Magnetic fields}
\lhead{\today}

\maketitle

\section*{Equations to learn}
\thispagestyle{empty}

\begin{itemize}
	
	\item The magnetic flux density is defined by:
	
	\[
		B = \frac{F}{IL}
	\]

	\item The magnetic force on a conductor carrying a current is:
	
	\[
		F=BIL
	\]

	\item However, if the direction of propogation of the current isn't
	perpendicular to the line between the North and South poles on the magnet,
	then the equation is:

	\[
		F=BIL\sin{\theta}
	\]

	\item The magnetic force on a moving charged particles is:

	\[
		F=BQv
	\]

	For electrons, the charge $Q$ is of course the elementary charge $e$ giving
	$F=Bev$

    \item We can combine $F=BQv$ with the equation for kinetic energy to create
	an equation to describe an electron travelling in a uniform magnetic field:

	\[
		\frac{mv^2}{r} = Bev
	\]

	\item The velocity of an undeflected charged particle in a region where
	electric and magnetic fields are at right angles is given by:

	\[
		v = \frac{E}{B}
	\]

\end{itemize}

\section*{Generating magnetic fields}

There are two ways to generate a magnetic field; bar magnets and an electrical
current. In both cases, the following rules apply:

\begin{itemize}

	\item Magnetic field lines come out of north poles and go into south poles.

    \item The direction of a field line at a point shows the direction of the
	force that a free magnetic {\bf north pole} would experience at that point.

	\item As with gravitational and electric fields, magnetic fields are
	strongest when the field lines are close together.

\end{itemize}

All magnetic fields are generated by moving charges. When a current flows in a
wire, the electrons move around the wire producing a magnetic field. In a bar
magnet, the magnetic field is produced by the electrons moving in their atoms.
As each electron orbits it's atom it creates a tiny magnetic field. In bar
megnets all of the electrons orbit in the same direction, so a net magnetic
field results.

\section*{Direction of fields}

Finding what direction a magnetic field is going in isn't tricky as long as you
remember three simple rules:

\begin{enumerate}

    \item {\bf The right hand grip rule} is used to find the direction of
	magnetic field lines in a wire. Simply form a 'thumbs up' shape with your
	hand, and point your thumb in the direction of the current in the wire.

	\item If you look at an electromagnet end on, you can see the coils going
	around, and you can work out which direction the current is flowing, either
	anticlockwise or clockwise. {\bf Clockwise is a south pole, anticlockwise
	is a north pole.}

	\item An alternative to the right hand grip rule is the {\bf right hand
	corkscrew rule}. To use this rule, imagine you are pushing a corkscrew into
	a cork, turning it as you do so. The direction in which you push is the
	direction of conventional current, and the direction in which you twist is
	the direction the field lines go in.

\end{enumerate}

\subsection*{Fleming's left hand motor rule}

We can use Fleming's left hand motor rule to predict the direction of the force
on a current carrying conductor. It is explained in the diagram below:

\begin{center}
	\includegraphics[scale=0.4]{left_hand_rule}
\end{center}

\section*{Magnetic flux density}

The density of magnetic field lines shows how strong the field is. Just like
gravitational and electrical fields, the field is strongest when the lines are
closest together.

The {\it strength} of the magnetic field is known as the {\bf magnetic flux
density}. The symbol for magnetic flux density is $B$ and the unit is the tesla
($T$).

The magnetic flux desnsity is defined in terms of the magnetic force experienced
by a current carrying conductor placed at right angles to a magnetic field. For
a uniform magnetic field, the flux density $B$ is defined by the equation:

\[
	B = \frac{F}{IL}
\]

Where $F$ is the force experienced by the current carrying conductor, $I$ is the
current in the conductor, $L$ is the length of the conductor in the uniform
magnetic field of flux density $B$. 

The direction of the force can be found by flemings left hand rule.

The tesla is defined as follows:

{\it The magnetic flux density is 1T when a wire carrying a current of 1A placed
at right angles to the magnetic field experiences a force of 1N per meter of
its length.}

When the field is at right angles to the current, we can use this equation to
find the force:

\[
	F = BIL
\]

\section*{Currents crossing fields}

It is important to remember that if a current carrying conductor is placed so
that current flows parallel to a magnetic field, then there will be no force on
the conductor.

\subsection*{At angles other than 90$^\circ$}

If the current carrying conductor is placed at an angle in the magnetic field,
then the force on the conductor is equal to:

\[
	F = BILsin\theta
\]

Note that this is an extention of what we already know. $sin\theta$ is equal to
zero when $\theta$ is zero, and is equal to one when $\theta$ is 90$^\circ$.

\section*{Which way is the current going?}

By convention, the direction of conventional electrical current is the direction
of flow of positive charge. When electrons are moving, the conventional current
is regarded as flowing in the opposite direction.

\section*{Magnetic force on a moving charge}

The magnetic force on a particle moving at right angles to a magnetic field is
given by the equation: \marginpar{\raggedright For an electron, the equation is
\newline $F=Bev$ where $e$ is the elementary charge}

\[
	F = BQv
\]

However this only gives us the magnitude of the force. The direction of the
force must be found using Fleming's left hand rule. We do know that the
direction of the force will be at 90$^\circ$ to the velocity.

If a charged particle moves at an angle $\theta$ to the magnetic field, we must
use the component of it's velocity to find the force:

\[
	F = BQvsin(\theta)
\]

\section*{Orbiting charges}

If a charged particle is moving in a uniform magnetic field, it will move in a
circle. This is because the magnetic force $F$ is always perpendicular to the
velocity, making it act as a centripetal force.

We can subsitute $F$ in the magnetic field equation for the formulea for
circular motion {\it centripetal force} $= \frac{mv^2}{r}$:

\[
	BQv = \frac{mv^2}{r}
\]

We can cancel off a $v$ and rearrange to get an equation to find the radius of
the orbit:

\[
	r = \frac{mv}{BQ}
\]

You can also find the momentum ($P$) of the particle: \marginpar{$P=mv$}

\[
	p = BQr
\]

From these equations, we can see that:

\begin{itemize}

	\item Faster moving particles have a bigger radius of orbit ($r \propto v$).

	\item Particles with greater masses have a bigger radius of orbit (since
	they have more momentum) ($r \propto m$).

	\item A stronger field will result in a smaller radius of orbit ($r \propto
	\frac{1}{b}$)

\end{itemize}

\section*{Electric and magnetic fields}

It is possible to make electrons move in straight lines by exposing them to both
an electrical and a magnetic field simultaneously. The electrostatic and
magnetic forces will be opposite, and with fine tuning, can become equal too.

The velocity of an undeflected charged particle in a region where electric and
magnetic fields are at right angles is given by:

\[
	v = \frac{E}{B}
\]

\end{document}
