\documentclass{article}

\usepackage[normalem]{ulem}
\usepackage{fancyhdr}
\usepackage[parfill]{parskip}
\usepackage{tikz}
\usepackage{pgfplots}
\usepackage{multicol}
\usepackage[version=3]{mhchem}
\usepackage{SIunits}
\usepackage{hyperref}
\usepackage{bookmark}
\makeatletter
\renewcommand\@seccntformat[1]{}
\makeatother

\newcommand{\parsec}{pc}

\pagestyle{fancyplain}

\pgfplotsset{compat=1.7}

\title{The nature of the universe}
\author{Todd Davies}
\date{\today}

\begin{document}

\rhead{The nature of the universe}
\lhead{\today}

\maketitle

\section{The cosmological principle}
\thispagestyle{empty}

The universe has been observed to be {\it homogenous}. This means that on a
large scale, it is the same at all places and has a constant density.

The universe is {\it isotropic}, meaning it is the same in all directions.
Cosmic microwave background radiation is the same in all directions, which
supports this theory.

The laws of physics are universal. This means that the laws of physics that
apply on Earth also apply on all other places in the universe.

The cosmological principle allows us to extrapolate our observations about one
part of the universe to the whole universe.

\section{The life of stars}

Most stars have a very long lifetime, often millions or billions of years. Since
we have not observed a whole lifecycle of any star we must rely on the fact that
the millions of stars that we can see represent all the stages of the life cycle
of a star.

\subsection{The Sun}

Stars such as the sun are formed from clouds of interstellar dust and gas, the
main components of which are hydrogen and helium. Such material is scattered
across the galaxy, but at dense points, its own gravity causes it to clump
together into a denser mass. This is called {\bf gravitational collapse}.

When the cloud of dust begins to contract, it loses potiential energy and gains
kinetic energy, which causes the cloud to heat up. At the center of the cloud,
the material is very dense and very hot. When the temperature is high enough
($\times10^7\kelvin$), fusion reactons begin to start.

The fusion of hydrogen is represented by the following nuclear equation:

\[
	4^1_1H \rightarrow ^4_2He + 2^{~0}_{+1}e + 2v
\]

The following reactions also happen:

\[
	^1_1H + ^1_1H \rightarrow ^2_1H + ^{~0}_{+1}e + v
\]

\[
	^1_1H + ^2_1H \rightarrow ^3_2He + \gamma
\]

\[
	^3_2H + ^3_2H \rightarrow ^4_2He + 2^{1}_{1}H
\]

Here, $\gamma$ represents a gamma-ray photon, and $v$ is a neutrino. The fusion
reaction releases energy (according to $E=mc^2$), which increases the
temperature of the star even more.

After a time, the star will reach a stable size and temperature. The energy
given off the star will equal the energy it generates, and the gravitational
force will equal the {\bf radiation pressure} from the photons emitted from the
star.

\subsection{The ageing Sun}

For any ageing star with a mass of less than three solar masses, the temperature
and pressure of the inner star increase, allowing more complex fusion reactions
to occur (producing elements such as silicon and iron). Eventually, these
reactions begin to slow down, and the core of the star begins to collapse under
gravitational attraction.

Now, the the layers of lighter nuclei (such as unburnt hydrogen and helium) on
the outer layers of the star start to fuse again (since the temperature of the
star has increased slightly). This causes the outer shell of the star to expand
due to radiation pressure. The size of the star increases and it's surface
temperature drops. The star is now a {\bf red giant}

As the core continues to collapse, the temperature there continues to increase,
making fusion happen phenomenally fast. This is called a helium flash and the
material surrounding the core is ejected away as a {\bf planetary
nebula}\marginpar{This doesn't create actual planets, it's just a legacy term.}.
What's left behind is the central core of the star, called a {\bf white dwarf}
that gradually cools and dims over millions of years.

\subsubsection{White dwarfs}

White dwarf stars have the following characteristics:

\begin{itemize}

	\item No fusion of hydrogen occurs. The star still glows because fusion
	reactions of the past are still leaking out of it.

	\item It is very dense. One teaspoon of white dwarf will have a mass of five
	tonnes.

    \item As the material of the star is so compressed, the electrons aren't
	attached to the atoms and move freely throughout the star. This is a state of
	matter called plasma.

	\item White dwarfs are prevented from further gravitational collapse by {\bf
	electron degeneracy pressure} This is because no two electrons may occupy
	the same quantum state, and a limit is reached in gravitational collapse
	where further collapse would require two or more electrons to exist in the
	same quantum state.

	\item The maximum mass of a white dwarf is about 1.4 solar masses. This is
	known as the Chandrasekhar limit.

\end{itemize}

\subsection{Large ageing stars}

Stars that have a mass of approximately three times larger than our sun (three
solar masses) swell to become super red giants when they come to the end of
their lives. When the star has its helium flash and collapses to become a white
dwarf, it's mass is still greater than 1.4 solar masses and so it collapses
further.

When this happens, the electrons in the star combine with protons to form
neutrons and neutrinos. The neutrinos escape and the core of the star is now
made entirely of closely packed neutrons. The outer layers of the star collapse
violently and rebound off the solid neutron core, generating a shock wave that
expolodes the outer layers of the star as a {\bf supernova}. Here, the heavy
elements in the star are blasted off into space.

Supernova events are rate (about one every 50 years in the milky way) but have
an intensity of around $\times10^{11}$ greater than a normal star, meaning that
they can be easily observed.

What remains of the core of the star after it has experienced a supernova
depends on its mass. For lighter stars, the core is made entirely of neutrons,
and is called a {\bf neutron star} with a very high density.

\subsubsection{Black holes}

For heavier stars, the supernova leaves a neutron star that is so massive that
it continues to collapse under its own gravity to form a black hole.

All the matter collapses to a point (a singularity). The gravitational field is
so strong that not even light can escape, meaning that it cannot be seen
directly, however black holes be observed to change the orbit of other stars
with their gravity.

\section{Measuring the universe}

Since the universe is so vast, SI units are usually too small for use when
describing quantities in the universe.

For measuring distances within our solar system, the {\bf astronomical unit}
(AU) is convinient. The astronomical unit is the average distance of the Earth
from the Sun.

\[
	1AU = 1.496\times10^{11}m \approx 1.5\times10^{11}m
\]

For measuring distances between stars in a galaxy, the {\bf light year} is
convinient. The light year is the distance travelled by light through a vacuum
in one year.

\[
	1ly \approx 9.46\times10^{15}m \approx 9.5\times10^{15}m 
\]

\subsection{The parsec}

Stars are seen to be in different positions in space depending on the position
of the Earth in its orbit. The stars are observed to move through an angle $2p$
during an orbit of the Earth around the Sun. The angle $p$ is the parallax of
the star. This angle is measured in {\it arc seconds}.

\[
	1 \textrm{ \it arc second} = \frac{1}{3600}degrees
\]

The parsec is defined as the distance that gives a parallax angle of one second.

\[
	Distance(pc) = \frac{1}{\textrm{\it parallax (arc seconds)}}
\]

\[
	\textrm{\it One parsec} \approx 3.1\times10^{16}m \approx 3.3ly
\]

\section{Olbers paradox}

Olbers paradox states that if the universe is infinate and uniform, then in
every direction we look from the Earth, ther will be a stare some distance away.
Even if the star is very far away and very dim, there will still be a star.
Concequently, the Earth ought to be bathed in an infinate amount of light.
However, it is obvious that this isn't the case.

Olbers paradox can be defined as:

{\it For an infinate, uniform and static universe, the night sky should be
bright because of light received in all directions from stars.}

\section{An expanding universe}

Vesto Slipher realised that the spectra produced telescopes trained on by
galaxies all had a similar pattern. He then realised that the lines
produced on spectra from different galaxies were shifted slightly out of
position (to the red or blue end of the spectrum).

\marginpar{\raggedright \it N.b. This equation only applies for galaxies
travelling siginificantly slower than the speed of light.}

The amount by which the spectra was shifted is called the {\bf redshift}.
Slipher then explained the cause of the redshift:

\begin{itemize}
	
	\item Electromagnetic waves that are emmited by a source that is moving away
	from the observer are streched out (according to the dopler effect).

	\item The greater the speed of the retreating object, the longer the waves
	are when they reach the observer, so the speed of the galaxies can be
	deduced. The following equation applies:

	\[
		\frac{\Delta \lambda}{\lambda} = \frac{v}{c}
	\]

\end{itemize}

\subsection{Hubble's law}

Edwin Hubble's greatest acheivment was to compare the distance of galaxies with
their speeds of recession. He deduced that:

\[
	\textrm{\it Speed of recession of galaxy} \propto 
	\textrm{\it Distance to galaxy}
\]

We can use Hubble's constant to turn this into an equation:

\[
	v = H_0x
\]

The measured value of $H_0$ is around
$70\kilo\meter\second^{-1}\mega\parsec^{-1}$

For example, a galaxy that is one megaparsec away will have a recession of
around $70kms^{-1}$.

The implication of Hubble's law is that the universe is expanding and galaxies
are moving away from each other. This is because all of the galaxies are moving
away from each other, they're not just moving away from us. The fabric of space
is expanding.

\subsection{The birth of the universe}

If the universe is expanding, then at some point in the past, it must have been
concentrated at one point. This is evidence the universe started from the big
bang.

If we plot a graph of the speed of the galaxies ($y$ axis) and their distance
apart ($x$ axis) (which is what Hubble did) the gradient is Hubble's constant.
Because all the galaxies originated from one point, the age of the universe must
be the reciprocal of the universe.

\[
	\textrm{\it Age of universe} \approx \frac{1}{H_0}
\]

\subsection{Converting units}

The age of the universe will only be in SI units (seconds) if Hubble's constant
is in SI units too. Hubble's constant is given as $70kms^{-1}Mpc^{-1}$, so we
need to convert parsecs to seconds and kilometers to meters.

\[
	\begin{split}
		1pc &\approx 3.1\times10^{16}m\\
		1Mpc &\approx 3.1\times10^{22}m\\
	\end{split}
\]

\[
	1km = 1\times10^{3}m
\]

Therefore:

\[
	\begin{split}
		70kms^{-1}Mpc^{-1} &= \frac{70kms^{-1}}{3.1\times10^{22}m}\\
						   &= \frac{70\times10^{3}ms^{-1}}{3.1\times10^{22}m}\\
						   &= 2.26\times10^{-18}s^{-1}
	\end{split}
\]

From this we can deduce the age of the universe:

\[
	\begin{split}
	\textrm{\it Age of universe} &\approx \frac{1}{2.26\times10^{-18}s^{-1}}\\
								 &\approx 4.43\times10^{17}s
	\end{split}
\]

This estimation is based on the assumption that the universe has been expanding
at a constant rate over the whole time it has existed.

\subsection{Revisiting Olvers paradox}

Because the universe is about $4.43\times10^{17}s$ or $14\times10^9$ years old,
we can only observe the universe that is within 14 billion light years of us.
This means the light from the most distant galaxies hasn't yet to reach us.

The universe isn't static, and probably isn't infinate. Also, as the distance
galaxies recede, their light will become redshifted and dimmer.

\end{document}