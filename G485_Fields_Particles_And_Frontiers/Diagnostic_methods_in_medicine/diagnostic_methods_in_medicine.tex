\documentclass{article}

\usepackage[normalem]{ulem}
\usepackage{fancyhdr}
\usepackage[parfill]{parskip}
\usepackage{tikz}
\usepackage{pgfplots}
\usepackage{multicol}
\usepackage[version=3]{mhchem}
\usepackage{SIunits}
\usepackage{lscape}
\usepackage{array}
\usepackage{hyperref}
\usepackage{bookmark}
\makeatletter
\renewcommand\@seccntformat[1]{}
\makeatother

\pagestyle{fancyplain}

\pgfplotsset{compat=1.7}

\title{16 - Diagnostic methods in medicine}
\author{Todd Davies}
\date{\today}

\begin{document}

\rhead{16 - Diagnostic methods in medicine}
\lhead{\today}

\maketitle

\section{Radioisotopes}

Radioactive isotopes of elements can be inserted into compounds that can be
ingested by/injected into a patient. This places a source of radiation inside
the patient, meaning that radiation will emerge from inside the body to produce
an image.

\subsection{Choosing a radionuclide}

There are a lot of different isotopes that emit radiation, but different
isotopes have different properties, meaning that they may be more or less
suitable for medical use.

First of all, the isotope used should emit only gamma rays. This ensures that
the radiation will penetrate the body so it can be detected by the detector (as
opposed to alpha or beta radiation). Gamma rays are also the least ionising of
the three types of radiation, which reduces the risk of cancer to the patient.

The half life of the isotope should have a fairly short half life so that it
will give out it's radiation quickly and it won't remain in the patient for a
long time (ensuring the patient and their environment isn't exposed to more
radiation than nececarry).

\subsubsection{Storing isotopes with short half lives}

How would a hospital be able to store isotopes with short half lives? They would
have to have shipments of the isotope every day or so in order to keep some
always in stock.

The most common solution to this problem is for the hospital to store a
radioactive isotope with a long half life that will decay into the desired
isotope with a short half life.

An example of this is that of $technetium-99m$:

\[
	\begin{split}
		^{99}_{42}Mo &\rightarrow ^{99}_{43}Tc^{m} + ^{\phantom{-}0}_{-1}e + \bar{\nu}\\
		^{99}_{43}Tc &\rightarrow ^{99}_{43}Tc + \gamma
	\end{split}
\]

The hospital will store some molybdenum which wil continually decay into
$^{99}_{43}Tc^{m}$. The 'm' indicates that the isotope is metastable, meaning
that it will release a gamma ray at some point. This element has a half life of
six hours, and when some needs to be used on a patient, it can be extracted from
the molybdenum and used.

\subsection{Radiopharmaceuticals}

In order to ensure that the radioactive isotope reaches the correct organ, it
must be formulated into a radiopharmaceutical. This means that it is chemically
combined with a substance that will be taken up by that organ.
Radiopharmaceuticals are sometimes called {\bf tracers} since they are designed
to targed particular areas of the body.

\section{The gamma camera}

A gamma camera is able to detect $\gamma$-rays from sources such as technetium-
99m.

Inside the camera is a collimator which consists of a honycomb of cylindrical
tubes, which stops photons with a horizontal component from entering any further
into the gamma camera.

Then, the gamma photons hit a large crystal called a scintillator, which acts to
convert one gamma photon into many visible photons.\marginpar{\raggedright This
works since gamma rays have a far higher energy than photons of visible light.}

\end{document}