\documentclass{article}

\usepackage[normalem]{ulem}
\usepackage{fancyhdr}
\usepackage[parfill]{parskip}
\pagestyle{fancyplain}

\title{Thermal physics}
\author{Todd Davies}
\date{\today}

\begin{document}

\rhead{Thermal physics}
\lhead{\today}

\maketitle

\section*{The kinetic model}
\thispagestyle{empty}
We assume some facts about particles when we're talking about kinetics:
\begin{itemize}
	\item Matter is made of tiny particles
	\item These particles tend to move around.
	\item Different states of matter represent particles doing different things. To describe states of matter on a microscopic scale, we use the following properties:
		\begin{itemize}
			\item The spacing between the particles
			\item The ordering of the particles
			\item The motion of the particles
		\end{itemize}
\end{itemize}

\section*{Brownian motion}
We can observe the random movement of the particles in gasses. Using larger visible particles in a transparent gas, we can see the motion of the large particles and so gain insight into the motion of the particles making up the gas. Brownian motion is the name given to the random movement of these visible particles. It provides evidence for:
\begin{itemize}
	\item Molecules in gasses move around...
	\item And do so randomly.
\end{itemize}

\section*{Explaining pressure}
Pressure is the result of the impulse exerted on a surface by molecules colliding with it. The force exerted on the surface from each collision is equal to:
\[
	F = \frac{\Delta p}{\Delta t} = \textrm{\textit{rate of change of momentum (= impulse)}}
\]
Since $\Delta p = mv - mu$ and the collision is elastic (so $mv = -mu$), we get:
\[
	F = -\frac{2mv}{\Delta t}
\]
The negative sign is present to show the force is exerted in the opposite direction to $v$.

From this equation, we can work out the macroscopic pressure on the surface:
\[
	pressure = \frac{\textrm{\textit{total force}}}{area}
\]

\section*{Changes of state}
When a substance is changing state, it is usually in the process of heating up or cooling down. However, when the substance is actually changing between states, the temperature of the substance is constant. This is because the energy that is being lost/gained by the substance is being used to create/break bonds between particles and increase/decrease the potential energy of the substance.

There are two terms you need to know for this bit:
\begin{itemize}
	\item \textbf{Latent heat of fusion}: \textit{The latent heat of fusion is the energy which must be supplied to cause a substance to melt at constant temperature}
	\item \textbf{Latent heat of vaporisation}: \textit{The latent heat of vaporisation is the energy which must be supplied to cause a substance to boil at constant temperature}
\end{itemize}
	
\subsection*{Evaporation}
Some gasses can exist below their boiling point. We use the term \textit{vapour} to describe them.

In a liquid, all the particles have different energies. The most energetic will be able to escape the liquid and boil off to form a vapour above the liquid. Since only the most energetic particles are able to escape, the ones left behind have a slightly lower average kinetic energy and so the overall temperature of the liquid falls.

\section*{Internal energy}
\textbf{Definition of internal energy}: \textit{The internal energy of a system is the sum of the ransom distribution of kinetic and potential energies of it's atoms or molecules.}

We can change the internal energy of a gas in two ways:
\begin{itemize}
	\item Heating a gas gives the particles more energy and makes them move faster.
	\item Doing work on a gas (e.g. compressing the gas) will increase the average kinetic energy of the molecules since the gas particles now collide with a moving wall and move off faster.
\end{itemize}

\end{document}