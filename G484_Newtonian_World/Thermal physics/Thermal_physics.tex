\documentclass{article}

\usepackage[normalem]{ulem}
\usepackage{fancyhdr}
\usepackage[parfill]{parskip}
\usepackage{multicol}
\pagestyle{fancyplain}

\title{Thermal physics}
\author{Todd Davies}
\date{\today}

\begin{document}

\rhead{Thermal physics}
\lhead{\today}

\maketitle

\section*{Equations to remember}
\begin{itemize}
	\item To convert between kelvin and calcius, use the following:
	\[
		T(K) = \Theta(^\circ C) + 273
	\]
	\item The equation for specific heat capacity is:
	\[
		SHC = \frac{\textrm{\textit{energy supplied}}}{mass \times \textrm{\textit{temperature change}}}
	\]
	Which as a 'proper' equation is:
	\[
		c = \frac{E}{m \Delta \Theta}
	\]
	\item The equation to find the energy change when changing the temperature of a substance is:
	\[
		E = mc \Delta \Theta
	\]
\end{itemize}

The quantities and their respective units are shown in this table:

\begin{center}
	\begin{tabular}{|l|l|l|}
		\hline
			Quantity & Abbreviation & Unit \\ \hline
			Absolute temperature & $K$ & Kelvin ($K$) \\ \hline
			Temperature & $^\circ C$ & Celsius ($^\circ C$) \\ \hline
			Specific heat capacity & $c$ & $Jkg^{-1}K^{-1}$ \\ \hline
			Energy & $E$ & Joules ($J$)\\ \hline
			Mass & $m$ & Kilograms ($kg$)\\ \hline
			Change in temperature & $\Delta \Theta$ & Kelvin ($K$) \\ \hline
	\end{tabular}
\end{center}

\section*{The kinetic model}
\thispagestyle{empty}
We assume some facts about particles when we're talking about kinetics:
\begin{itemize}
	\item Matter is made of tiny particles
	\item These particles tend to move around.
	\item Different states of matter represent particles doing different things. To describe states of matter on a microscopic scale, we use the following properties:
		\begin{itemize}
			\item The spacing between the particles
			\item The ordering of the particles
			\item The motion of the particles
		\end{itemize}
\end{itemize}

\section*{Brownian motion}
We can observe the random movement of the particles in gasses. Using larger visible particles in a transparent gas, we can see the motion of the large particles and so gain insight into the motion of the particles making up the gas. Brownian motion is the name given to the random movement of these visible particles. It provides evidence for:
\begin{itemize}
	\item Molecules in gasses move around...
	\item And do so randomly.
\end{itemize}

\section*{Explaining pressure}
Pressure is the result of the impulse exerted on a surface by molecules colliding with it. The force exerted on the surface from each collision is equal to:
\[
	F = \frac{\Delta p}{\Delta t} = \textrm{\textit{rate of change of momentum (= impulse)}}
\]
Since $\Delta p = mv - mu$ and the collision is elastic (so $mv = -mu$), we get:
\[
	F = -\frac{2mv}{\Delta t}
\]
The negative sign is present to show the force is exerted in the opposite direction to $v$.

From this equation, we can work out the macroscopic pressure on the surface:
\[
	pressure = \frac{\textrm{\textit{total force}}}{area}
\]

\section*{Changes of state}
When a substance is changing state, it is usually in the process of heating up or cooling down. However, when the substance is actually changing between states, the temperature of the substance is constant. This is because the energy that is being lost/gained by the substance is being used to create/break bonds between particles and increase/decrease the potential energy of the substance.

There are two terms you need to know for this bit:
\begin{itemize}
	\item \textbf{Latent heat of fusion}: \textit{The latent heat of fusion is the energy which must be supplied to cause a substance to melt at constant temperature}
	\item \textbf{Latent heat of vaporisation}: \textit{The latent heat of vaporisation is the energy which must be supplied to cause a substance to boil at constant temperature}
\end{itemize}
	
\subsection*{Evaporation}
Some gasses can exist below their boiling point. We use the term \textit{vapour} to describe them.

In a liquid, all the particles have different energies. The most energetic will be able to escape the liquid and boil off to form a vapour above the liquid. Since only the most energetic particles are able to escape, the ones left behind have a slightly lower average kinetic energy and so the overall temperature of the liquid falls.

\section*{Internal energy}
\textbf{Definition of internal energy}: \textit{The internal energy of a system is the sum of the ransom distribution of kinetic and potential energies of it's atoms or molecules.}

We can change the internal energy of a gas in two ways:
\begin{itemize}
	\item Heating a gas gives the particles more energy and makes them move faster.
	\item Doing work on a gas (e.g. compressing the gas) will increase the average kinetic energy of the molecules since the gas particles now collide with a moving wall and move off faster.
\end{itemize}

\section*{Calculating energy changes}
\subsection*{Specific heat capacity}
\textbf{Definition:}\textit{The specific heat capacity of a substance is the energy required per unit mass of the substance to raise the temperature by 1K.}

The energy required to heat an object depends on three things:
\begin{itemize}
	\item The mass of the material
	\item The change in temperature we are trying to achieve
	\item The SHC of the material
\end{itemize}

We can represent this as an equation:
\[
	E = mc \Delta \Theta
\]
or we can rearrange it to get:
\[
	c = \frac{E}{m \Delta \Theta}
\]
From this equation, you can work out the unit of specific heat capacity - $J kg^{-1} K^{-1}$

Substances with a low SHC are able to change temperature very quickly and so the gradient of a temperature/time graph is usually steep.

\subsection*{Finding the specific heat capacity of a substance}
To find the SHC of a material, all we have to do is supply a known amount of energy to a substance and monitor it's temperature change.

Using an electric heater (with a known and constant current and voltage) we can supply a known amount of energy ($E$) to the substance. We must then find the difference in temperature before and after the energy has been supplied ($\Delta \Theta$). We can weigh the object to find it's mass and then sub all these values into the SHC equation to get our final answer.

An alternate way of finding the SHC would be to (again) use an electric heater to supply a known amount of heat energy to the object and then plot a temperature/time graph. Then we divide both sides of the SHC equation by $\Delta t$:
\[
	\frac{E}{\Delta t} = mc\frac{\Delta \Theta}{\Delta t}
\]
Since $\frac{E}{\Delta t}$ is equal to the power supplied by the heater, we can work out the heater's power and sub that in:
\[
	P = VI
\]
\[
	VI = mc\frac{\Delta \Theta}{\Delta t}
\]
We can re-arrange this equation to find the SHC:
\[
	c = \frac{VI \times \Delta t}{m \Delta \Theta}
\]


\subsubsection*{Sources of error}
\begin{itemize}
	\item It is desirable to have a low rate of heating since then the heat spreads evenly throughout the material.
	\item The object must be thermally insulated. Some heat \textit{will} escape to the surroundings making obtained value for the SHC an overestimate.
\end{itemize}

\subsection*{Specific latent heat}
\textbf{Definition:} \textit{The specific latent heat of a substance is the energy required per kilogram of the substance to change it's state without any change in temperature.}

The specific \textbf{latent heat of fusion} is for \textbf{melting} and the \textbf{specific latent heat} of \textbf{vaporisation} is for boiling.

\section*{Questions to practice}
\textbf{Question 1}\\
Explain why if you leave a pan on  the hob for a long time and the water reaches 100$^{\circ}C$, it doesn't all boil away at once.

\textbf{Answer}\\
Even though the average temperature of all the molecules is 100$^{\circ}C$, not all will be hot enough to evaporate. Even though the temperature is 100$^{\circ}C$ - the boiling point, it takes more energy to separate the molecules to form steam.

\textbf{Question 2}\\
Calculate the energy required to raise the temperature of 5kg of water from 20$^{\circ}C$ to 100$^{\circ}C$. (SHC of water = 4180 $Jkg^{-1}K^{-1}$)

\textbf{Answer}\\
\[
	E = mc \Delta \Theta = 5 \times 4180 \times 80 = 1.7MJ
\]

\newpage

\textbf{Question 3}\\
An insulated block of mass $1.2kg$ is heated by a $50W$ heater for four minutes. The temperature change is from 22$^{\circ}C$ to 45$^{\circ}C$. Find the SHC of the block.

\textbf{Answer}\\
\[
	c = \frac{E}{m \Delta \Theta} = \frac{Pt}{m \Delta \Theta} = \frac{50 \times (4*60)}{1.2 \times (45-22)} = 435 Jkg^{-1}K^{-1}
\]

\textbf{Question 4}\\
A student measures the SHC of water. This is the data from his experiment:
\begin{multicols}{2}
	\begin{itemize}
		\item	Mass of beaker = 150g
		\item Mass of water = 522g
		\item Current in the heater = 3.9A
		\item p.d. across the heater = 11.4V
		\item Initial temperature = 18.5$^{\circ}C$
		\item Final temperature = 30.2$^{\circ}C$
		\item Time taken = 13 minuets
	\end{itemize}
\end{multicols}

\textbf{Answer}\\
\[
	c = \frac{VIt}{m \Delta \Theta} = \frac{3.9 \times 11.4 \times (13 \times 60)}{0.522 \times (30.2-18.5)} = 5678.2 Jkg^{-1}K^{-1}
\]


\end{document}