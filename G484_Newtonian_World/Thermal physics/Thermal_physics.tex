\documentclass{article}

\usepackage[normalem]{ulem}
\usepackage{fancyhdr}
\usepackage[parfill]{parskip}
\pagestyle{fancyplain}

\title{Thermal physics}
\author{Todd Davies}
\date{\today}

\begin{document}

\rhead{Thermal physics}
\lhead{\today}

\maketitle

\section*{The kinetic model}
\thispagestyle{empty}
We assume some facts about particles when we're talking about kinetics:
\begin{itemize}
	\item Matter is made of tiny particles
	\item These particles tend to move around.
	\item Different states of matter represent particles doing different things. To describe states of matter on a microscopic scale, we use the following properties:
		\begin{itemize}
			\item The spacing between the particles
			\item The ordering of the particles
			\item The motion of the particles
		\end{itemize}
\end{itemize}

\section*{Brownian motion}
We can observe the random movement of the particles in gasses. Using larger visible particles in a transparent gas, we can see the motion of the large particles and so gain insight into the motion of the particles making up the gas. Brownian motion is the name given to the random movement of these visible particles. It provides evidence for:
\begin{itemize}
	\item Molecules in gasses move around...
	\item And do so randomly.
\end{itemize}

\section*{Explaining pressure}
Pressure is the result of the impulse exerted on a surface by molecules colliding with it. The force exerted on the surface from each collision is equal to:
\[
	F = \frac{\Delta p}{\Delta t} = \textrm{\textit{rate of change of momentum (= impulse)}}
\]
Since $\Delta p = mv - mu$ and the collision is elastic (so $mv = -mu$), we get:
\[
	F = -\frac{2mv}{\Delta t}
\]
The negative sign is present to show the force is exerted in the opposite direction to $v$.

From this equation, we can work out the macroscopic pressure on the surface:
\[
	pressure = \frac{\textrm{\textit{total force}}}{area}
\]
\end{document}