\documentclass{article}

\usepackage[normalem]{ulem}
\usepackage{fancyhdr}
\usepackage[parfill]{parskip}
\pagestyle{fancyplain}

\title{Oscillations}
\author{Todd Davies}
\date{\today}

\begin{document}

\rhead{Oscillations}
\lhead{\today}

\maketitle

\section*{What's an oscillation?}
\thispagestyle{empty}
An oscillation is defined as a '\textit{A repetitive back and forth motion}' (an object that oscillates can also be said to vibrate).

\section*{Types of oscillation}
There are two types of oscillation - free and forced.
\subsection*{Free oscillations}
This is when an object is left to vibrate at it's \textit{natural frequency}. The oscillating object is under no external influence (other than the influence that initiated the motion). 

An example of a free oscillation could be when you pluck a violin string. The string oscillates at its natural frequency which produces the note that you hear.
\subsection*{Forced oscillations}
A forced oscillation is when an object moves back and forth under the influence of another driving object.

An example of this is if you wave your hand at somebody. Your arm is driving your hand, but your hand isn't moving at it's natural frequency.
\end{document}