\documentclass{article}

\usepackage{fancyhdr}
\usepackage{array}
\usepackage{longtable}
\usepackage{booktabs}
\usepackage{calc}
\pagestyle{fancyplain}

\title{Definitions \\ \large{G484 - Newtonian World}}
\author{Todd Davies}
\date{\today}

\begin{document}

\rhead{Definitions}
\lhead{\today}

\maketitle

\thispagestyle{empty}

\begin{longtable}{>{\bf\centering\arraybackslash}p{1in} 
  p{\textwidth-4\tabcolsep-1in}}

\large{Name} & \large{\textbf{Definition}}\\ \midrule
\endfirsthead
\large{Name} & \large{\textbf{Definition}}\\ \midrule
\endhead
\midrule
\multicolumn{2}{r}{continued \ldots}
\endfoot
\endlastfoot
  Newton's first law & An object will remain at rest or keep 
    travelling at a constant velocity unless it is acted upon by an external 
    force.\\ \midrule
  Newton's second law & The net force acting on an object is equal
    to the rate of change of it's momentum. The net force and change in momentum
    are in the same direction.\\ \midrule
  Newton's third law & When two bodies interact, the forces they
    exert on each other are equal and opposite.\\ \midrule
  Linear momentum & Linear momentum is the product of an object's mass and
    velocity ($p=mv$). Momentum is a vector quantity.\\ \midrule
  Net force on a body & The rate of change of the momentum of the body
  (also known as Impulse).\\ \midrule
  Impulse of a force & The product of the force ($F$) and the time ($\Delta t$)
    for which it acts ($impulse = F \Delta t$).\newline \newline This is the
    area under a force time graph.\newline \newline Impulse = change in momentum
    ($impulse = \Delta p$)\\ \midrule
  Principle of conservation of momentum & In a closed system, when bodies
    interact, the total momentum in any specified direction remains constant.\\
    \midrule
  Perfectly elastic collision & A collision is perfectly elastic
    when kinetic energy is conserved. Momentum and total energy are always
    conserved.\\ \midrule
  Inelastic collision & A collision is inelastic when the kinetic
    energy is not conserved, and some is transferred to other forms such as
    heat. Momentum and total energy are always conserved.\\ \midrule
  Radian & $\pi$ radians = $180^\circ$\\ %\midrule
  Centripetal force & The net force acting on an object moving in a circle. It 
    is always directed towards the center of the circle. \\ \midrule
  Gravitational field strength & The gravitational force 
    experienced by an object per unit mass ($g = \frac{F}{m}$).\\ \midrule
  Newton's law of gravitation & Any two point masses attract each other with a
    force that is directly proportional to the square of their masses and
    inversely proportional to the square of their seperation. ($F = 
    -\frac{GMm}{r^2}$)\\ \midrule
  Period of an object & The time taken by any object (e.g. a planet) to complete
    an one orbit. \newline \newline The period is also the time taken for one
    complete oscillation of a vibrating object.\\ \midrule
  Geostationary orbit & The orbit of an artificial satellite which has a period
    equal to one day so that the satellite remains above the same point on the
    Earth's equator. \newline \newline From Earth, the satellite appars to be
    stationary.\\ \midrule
  Displacement & Displacement is the distance moved by an object from a fixed
    starting point.\\ \midrule
  Amplitude & The maximum displacement of a particle from it's equlibrium
    position.\\ \midrule
  Frequency & The number of oscillations of a particle per unit time.\\ \midrule
  Angular frequency & The rate of change of an angle (expressed in radian per 
    second). \newline \newline Angular frequency $\omega = \frac{2\pi}{T}$\\ 
    \midrule
  Phase difference & The fraction of an oscillation between the vibrations of
    two oscillating particles, exppressed in degrees or radians.\\ \midrule
  Simple harmonic motion (SHM) & Motion of an oscillator where its acceleration
    is directly proportional to its displacement from its equlibrium position
    and is directed towards that position.\\
  Conditions for simple harmonic motion & A mass must oscillate. \newline
    \newline There must be a position where the mass is in equlibrium.\newline
    \newline A restoring force that acts to return the mass to its equlibrium
    position. The restoring force is directly proportional to the displacement
    of the mass from its equlibrium posision and is directed towards that point.
    \\ \midrule
  Pressure & The force acting on a surface per unit area.\\ \midrule
  Internal energy & The sum of the random distributions of kinetic and
    potiential energies of the atoms or molecules in a system.\\ \midrule
  Specific heat capacity & The energy required per unit mass of a substance to
    change its state without any change in temperature.\newline \newline
    Unit is $Jkg^{-1}$\\ \midrule
  Latent heat of fusion & The energy absorbed by a substance to change state
    from solid to liquid without any change in temperature. \\ \midrule
  Latent heat of vaporisation & The energy absorbed by a substance to change
    state from liquid to gas without any change in temperature. \\ \midrule
  Boyle's law & The pressure exerted by a fixed mass of gas is inversely
    proportional to its volume, provided the temperature of the gas remains
    constant.\\ \midrule
  Assumptions of the kinetic theory of gasses & A gas contains a very large
    number of spherical particles.\newline \newline The forces between particles
    are negligible, except during collisions.\newline \newline The volume of the
    particles is negligible compared to the volume occupied by the gas.\newline
    \newline Most of the time, a particle moves in a straight line at a constant
    velocity. The time of collision with each other or with the container walls
    is negligible compared with the time between collisions.\newline \newline
    The collisions of particles with each other and with the container are 
    perfectly elastic, so that no kinetic energy is lost.\\ \midrule
\end{longtable}

\end{document}