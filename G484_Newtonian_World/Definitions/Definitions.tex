\documentclass{article}

\usepackage{fancyhdr}
\usepackage{array}
\usepackage{tabularx}
\usepackage{pbox}
\pagestyle{fancyplain}

\title{Definitions \\ \large{G484 - Newtonian World}}
\author{Todd Davies}
\date{\today}

\begin{document}

\rhead{Definitions}
\lhead{\today}

\maketitle

\thispagestyle{empty}

%\begin{tabular}{>{\bf}l | l}
%\large{Name} & \large{\textbf{Definition}}
%\end{tabular}
\begin{tabularx}{\textwidth}{>{\bf\centering\arraybackslash}m{1in} | X}
  \large{Name} & \large{\textbf{Definition}}\\ \hline
  Newton's first law & An object will remain at rest or keep 
  	travelling at a constant velocity unless it is acted upon by an external 
  	force.\\ \hline
  Newton's second law & The net force acting on an object is equal
  	to the rate of change of it's momentum. The net force and change in momentum
  	are in the same direction.\\ \hline
  Newton's third law & When two bodies interact, the forces they
  	exert on each other are equal and opposite.\\ \hline
  Linear momentum & The product of an object's mass and velocity
  	($p=mv$). Momentum is a vector quantity.\\ \hline
  Net force on a body & Is said to be equal to the rate of change
  	of the momentum of the body (Impulse).\\ \hline
  Impulse of a force & The product of the force ($F$) and the time ($\Delta t$)
  	for which it acts ($impulse = F \Delta t$).\newline \newline This is the
  	area under a force time graph.\newline \newline Impulse = change in momentum
  	($impulse = \Delta p$)\\ \hline
  Principle of conservation of momentum & In a closed system, when bodies
  	interact, the total momentum in any specified direction remains constant.\\
  	\hline
  Perfectly elastic collision & A collision is perfectly elastic
  	when kinetic energy is conserved. Momentum and total energy are always
  	conserved.\\ \hline
  Inelastic collision & A collision is inelastic when the kinetic
  	energy is not conserved, and some is transferred to other forms such as
  	heat. Momentum and total energy are always conserved.\\ \hline
  Radian & $\pi$ radians = $180^\circ$\\ \hline
  Gravitational field strength & The gravitational force 
  	experienced by an object per unit mass ($g = \frac{F}{m}$).\\ \hline
  Newton's law of gravitation & Any two point masses attract each other with a force that is directly proportional to the square of their masses and inversely proportional to the square of their seperation\\ \hline
\end{tabularx}

\end{document}