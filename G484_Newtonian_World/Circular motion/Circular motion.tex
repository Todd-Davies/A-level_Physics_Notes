\documentclass{article}

\usepackage[normalem]{ulem}
\usepackage{fancyhdr}
\usepackage[parfill]{parskip}
\pagestyle{fancyplain}

\title{Circular motion}
\author{Todd Davies}
\date{\today}

\begin{document}

\rhead{Circular motion}
\lhead{\today}

\maketitle

\section*{What is circular motion?}
\thispagestyle{empty}
Circular motion is a movement of an object along the circumference of a circle
or rotation along a circular path. Some examples of circular motion include:
\begin{itemize}
	\item Wheels of a car.
	\item Orbit of planets around the Sun.
	\item The hands of a clock.
\end{itemize}

\section*{Equations to remember}
\label{sec:Equations To Remember}
The equations in this topic aren't very hard to remember and there isn't many of
them. However, the concepts required to understand them are sometimes more
difficult to grasp.

\begin{itemize}
	\item To convert between radians and degrees, use the following:
	\[
		radians = \frac{2\pi}{360} \times degrees
	\]
	\item The time taken for an object to go around a circle once is called the
	time period. It's relation to the frequency of the motion is defined by:
	\[
		f = \frac{1}{T}
	\]
	\item The centripetal force acting on an object of a given mass and
	tangential velocity moving in a circular motion with a given radius can be
	found using the equation:
	\[
		F = \frac{mv^2}{r}
	\]
	\item The centripetal acceleration of an object with a given tangential
	velocity and moving in a circular motion with a given radius can be
	calculated by using:
	\[
		a = \frac{v^2}{r}
	\]
	\textit{N.b. This can also be used to find the force of gravity on an object
	(since you can find the acceleration due to gravity).}
	\item To find the speed of an object travelling around a circle:
	\[
		v = \frac{2 \pi r}{t}
	\]
	\textit{You might have noticed, this is just a slight variation of speed
	equals distance over time.}
	\item The angular velocity of an object is equal to the angle rotated
	divided by time:
	\[
		\omega = \frac{ \theta }{t}
	\]
\end{itemize}

The quantities and their respective units are shown in this table:

\begin{center}
	\begin{tabular}{|l|l|l|}
		\hline
			Quantity & Abbreviation & Unit \\ \hline
			Time period & $T$ & Second ($s$) \\ \hline
			Frequency & $f$ & Hertz ($Hz$) \\ \hline
			Force & $F$ & Newton ($N$) \\ \hline
			Mass & $m$ & Kilogram ($kg$) \\ \hline
			Radius & $r$ & Metre ($m$) \\ \hline
			Acceleration & $a$ & Metres per second squared ($ms^{-2}$) \\ \hline
			Tangential velocity & $v$ & Metres per second ($ms^{-1}$) \\ \hline
			Time & $t$ & Seconds ($s$) \\ \hline
			Angular velocity & $\omega$ & Radians per second ($rad/s$) \\ \hline
			Angle & $\theta$ & Radians ($rad$) \\ \hline
	\end{tabular}
\end{center}


\section*{Points to remember}

\begin{itemize}
	\item Objects moving in a circle aren't in equilibrium. They need an
	external force to act on them to keep them moving in a circle. (Think
	Newton's first law - an object remains in equilibrium (i.e. moving at a
	constant speed in a straight line) unless acted upon by another force).
	\item The centripetal force is an imaginary force we use to describe all the
	forces that add up together to make an object move in a circle. It's
	\textbf{not} an actual force.
	\item The centripetal force always acts towards the centre of the circle,
	and since the velocity of the object is a tangent on the circular path of
	the object, the centripetal force is also perpendicular to the velocity of
	the object.
	\item Objects travelling in a circular motion may (not always) have a steady
	speed, but their velocity is constantly changing. This is because velocity
	is a vector quantity - it has a magnitude and a direction while speed is a
	scalar quantity, having only a magnitude.
	\item The centripetal force (like all forces) has both a horizontal and
	vertical component. You can use this to explain situations where objects
	'defy' gravity. For example, if you whirl a conker around your head, the
	vertical component of the tension in the string is equal to the weight of
	the conker.
\end{itemize}


\section*{Questions to practice}
\textbf{Question 1}\\
Calculate the magnitude of the centripetal force that keeps the Earth in orbit
around the Sun.
Where:
\begin{center}
	\begin{tabular}{ l l }
		Mass of the Earth & $6.0\times10^{24}kg$ \\
		Speed of the Earth in orbit & $30000ms^{-1}$ \\
		Radius of the orbit of the Earth around the Sun & $1.5\times10^{11}m$
	\end{tabular}
\end{center}
\textbf{Answer}\\
We can use the centripetal force equation to find the force like so:
\[
	F = \frac{6.0\times10^{24} \times 30000^2}{1.5\times10^{11}}
\]
\[
	F = \uline{3.6 \times 10^{22}N}
\]

\textbf{Question 2}\\
What provides the centripetal force that keeps the Earth in orbit?

\textbf{Answer} \\
The gravity of the Sun provides the centripetal force keeping Earth orbiting it.

\textbf{Question 3}\\
If you were to swing a bucket of water in vertical circle with a radius of 1m
what is the minimum time the bucket could take to complete the circle without
spilling water?

\textbf{Answer} \\
First, we must find the minimum velocity that the bucket can travel without
having water fall out, then we can find how long it'll take for the bucket to
travel the circumference of the circle at this speed.

We can use $F = \frac{mv^2}{r}$ along with $F = ma$ to find the minimum velocity
that we can have the bucket travel at without spillage. Combining the two
equations gives us:
\[
	\frac{mv^2}{r} = ma
\]
The masses cancel out and we can sub in the the value for gravity as
acceleration ($9.8ms^{-2}$) and the radius ($1m$) so $\frac{v^2}{1} = 9.8$.

Therefore 
\[
	v = 3.13ms^{-1}
\]
Now we have to find the time it takes for the the bucket to travel around the
circumference of the circle, so we must first find it's circumference: $radius
\times 2\pi$ Therefore $radius = 1 \times 2\pi = 2\pi$

To find the minimum time, we must divide the circumference by the velocity: 
	\[
		\frac{2\pi}{3.13} = \uline{2.01s}
	\]
\textbf{Question 4}\\
Explain why an aeroplane will lose altitude when banking unless the pilot
increases the throttle.

\textbf{Answer}\\
When the aeroplane is banking, the upwards lift = $\cos\theta \times$total lift.
When $0 < \theta < 90$ (i.e. when the plane is banking) $cos\theta<1$ so the
pilot must increase the total lift to keep the same vertical lift.

\textbf{Question 5}\\
A lawnmower rotates at $3500rpm$. The blade has a radius of $0.23m$. Find the
velocity and acceleration at the tip of the blade.

\textbf{Answer}\\
To solve this question we must first convert rpm into rps:
\[
	3500rpm = \frac{3500}{60}rps = 58.3rps
\]
Then we can work out the time period of the blade.
\[
	t = \frac{1}{f} = \frac{1}{58.3} = 0.01714...s
\]
Then we can sub it into the equation to find the speed of an object travelling
around a circle:
\[
	v = \frac{2 \pi r}{t} 
	  = \frac{2 \times \pi \times 0.23}{0.01714} 
	  = \uline{84.3m/s}
\]
Now we can work out the acceleration:
\[
	a = \frac{v^2}{r} = \frac{84.3^2}{0.23} = \uline{30897ms^{-2}}
\]

\textbf{Question 6}\\
A pendulum is swinging. It swings up to 90 degrees and back again every time. 
Find the tension in the string when the pendulum is at the bottom of it's swing.
The length of the string is $h$.

\textbf{Answer}\\
The gravitational potential energy is equal to the kinetic energy, so we can 
find $v$ in terms of the length of the string (since the height of the swing is
the radius of the swing (since it swings to 90 degrees)):
\[
	mgh = \frac{1}{2}mv^2
\]
\[
	v = \sqrt{2gh}
\]
Now we need to find the tension.
\[
	F = \frac{mv^2}{r}
\]
Since the centripetal force at the bottom of the swing is equal to the tension
minus the weight:
\[
	T - mg = \frac{mv^2}{h}
\]
Now we can sub in out value for $v$:
\[
	T - mg = \frac{m({\sqrt{2gh}})^2}{h} = \frac{2mgh}{h} = 2mg
\]
\[
	T = \uline{3mg}
\]
\end{document}