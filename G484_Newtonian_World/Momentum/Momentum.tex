\documentclass{article}

\usepackage[normalem]{ulem}
\usepackage{fancyhdr}
\usepackage[parfill]{parskip}
\pagestyle{fancyplain}

\title{Momentum}
\author{Todd Davies}
\date{\today}

\begin{document}

\rhead{Momentum}
\lhead{\today}
\setlength{\parindent}{0cm}

\maketitle


\section*{What is momentum?}
\thispagestyle{empty}
\label{sec:WhatIsMomentum}
Momentum is the quantity of motion of a moving body. It is measured as the product of the mass and the velocity of the body.

The idea of momentum can be used to explain and predict the outcome of collisions. It often comes up in questions about space too.


\section*{Equations to remember}
\label{sec:Equationstoremember}
The trickiest thing about momentum equations is remembering to take negative values into account. If one object is travelling one way and another is travelling the opposite way, then if they collide, the final momentum will be equal to the first object's momentum take the second object's momentum.


\begin{itemize}
	\item The equation for momentum is:
	\[
		p = mv
	\]
	\item The total momentum in a closed system is constant.
	\[
		m_1 u_1 + m_2 u_2 = m_1 v_1 + m_2 v_2
	\]
	\item The equation to calculate the kinetic energy of an object is:
	\[
		KE = \frac{1}{2}mv^2
	\]
\end{itemize}

The quantities and their respective units are shown in this table:

\begin{center}
	\begin{tabular}{|l|l|l|}
		\hline
			Quantity & Abbreviation & Unit \\ \hline
			Momentum & $p$ & Kilogram meter per second($kgms^{-1}$) \\ \hline
			Mass & $m$ & Kilogram($kg$) \\ \hline
			Velocity & $v$ & Meter per second ($ms^{-1}$) \\ \hline
			Kinetic energy & $KE$ & Joules ($J$) \\ \hline
	\end{tabular}
\end{center}

\section*{Collisions}
\label{sec:Collisions}
The law of conservation of momentum states that in a closed system, the total momentum in a system is constant. This law is key to working out the momentum of objects before and after collisions.

The total energy in a closed system is also conserved during a collision.


\subsection*{Elastic collisions}
\label{sec:ElasticCollisions}
In an elastic collision, no kinetic energy is converted to other forms of energy.
Very few collisions are \textit{truly} elastic, since energy is nearly always lost as sound/heat etc. In fact it is only really at the atomic level, do we see elastic collisions since no kinetic energy can be converted to heat or sound energy as at this level, heat and sound energy \textit{are} kinetic energy!

The term 'springy collision' is used to describe a collision that is nearly elastic. An example of this would be a stationary snooker ball being hit by a moving ball. The ball that is initially stationary would gain the momentum of the ball that was initially moving, and in order to obey the law of conservation of momentum, the ball that was moving would now be stationary.


\subsection*{Inelastic collisions}
\label{sec:InelasticCollisions}
When two objects collide, sometimes they will deform or even stick together. There could be blue-tack on them, they could have opposing (and mutually attractive) charges, they could deform or even physically join.

If this happens, the law of conservation of momentum still applies, and so you can calculate the momentum of the objects before and after the collision.

However, kinetic energy isn't conserved during these collisions, since energy is often lost as heat, is used deforming the objects that collide.


\section*{Explosions}
\label{sec:Explosions}
Explosions can be treated in the same way that collisions can, the total momentum in the system is constant. This explains why a firework produces a circular pattern in the sky when it explodes. As it is flying up, the momentum of the firework is going upwards. When it explodes, the total momentum is still going upwards, and though each particle ejected has a different direction of motion they all add up to the same as the momentum before the explosion.

Another commonly used example is of a astronaut on a space-walk. If the astronaut becomes too far away from the space ship, he can throw one of his tools in the opposite direction. Since the tool gains momentum and is travelling away from the ship, the astronaut will also gain momentum in the opposite direction towards the space ship.


\section*{Practice questions}
\label{sec:PracticeQuestions}
\textbf{Question 1}\\
A skinny man and a fat man have a belly bounce competition. The fat man has a mass of $140kg$ and the skinny man has a mass of $75kg$. They are running at speeds of $6ms^{-1}$ and $10ms^{-1}$ respectively. After the impact, the fat man bounces back with a velocity of $2ms^{-1}$. Find the final velocity of the skinny man.

\textbf{Answer} \\
Sub the values into the equation for conservation of momentum to find the final momentum of the skinny man:
\[
	m_1 u_1 + m_2 u_2 = m_1 v_1 + m_2 v_2
\]
\[
	140 \times 6 + 75 \times 10 = 140 \times 2 + 75 \times ?
\]
\[
	1590 = 280 + ?
\]
\[
	1310 kgms^{-1}
\]
Now we can find final velocity of the skinny man:
\[
	p = mv
\]
\[
	1310 = 75 \times v
\]
\[
	v \approx \uline{15.5ms^{-1}}
\]

\textbf{Question 2}
Now find the change in kinetic energy of the skinny man.

\textbf{Answer}
Find the kinetic energy of the skinny man before and after the collision and take one off the other to find the change in kinetic energy:
\[
	\Delta KE = \frac{1}{2}m_1 \times {v_1}^2 - \frac{1}{2}m_2 \times {v_2}^2
\]
\[
	\Delta KE = \frac{1}{2}75 \times 10^2 - \frac{1}{2}75 \times 15.5^2
\]
\[
	\Delta KE = 3750 - 9009
\]
\[
	\Delta KE = \uline{5259J}
\]

\textbf{Question 3}
Define \textit{linear momentum}.

\textbf{Answer}
Linear momentum is the product of the mass and the velocity of an object.
\end{document}