\documentclass{article}

\usepackage[normalem]{ulem}
\usepackage{fancyhdr}
\usepackage[parfill]{parskip}
\pagestyle{fancyplain}

\title{Gravitational fields}
\author{Todd Davies}
\date{\today}

\begin{document}

\rhead{Circular motion}
\lhead{\today}
\setlength{\parindent}{0cm}

\maketitle

\section*{What is gravity?}
\thispagestyle{empty}
Gravity is the attraction between all matter and is one of the four fundamental forces of the universe. Large objects such as planets have a very strong gravity, while smaller objects have a weaker gravity. Gravity has an inverse square relationship between strength and distance, the closer two objects get to each other, the larger the force between them.

Gravity can be represented by field lines. These have arrows that show the direction of the gravitational field on a mass placed in the field, and the spacing of the lines indicates the strength of the field (a more dense spacing corresponds to a stronger field).

We can treat most objects as a point mass when we're working with them in relation to the gravitational fields they produce and interact with. The point is centred around the centre of gravity of the object.


\section*{Equations you need to know}

\begin{itemize}
	\item The gravitational field strength at a point is equal the the gravitational force exerted per unit mass on an object placed at the point:
	\[
		g = \frac{F}{m}
	\]
	\item The equation for Newton's law of gravitation states is:
	\[
		F = -\frac{GMm}{r^2}
	\]
	\item The gravitational field strength at a specified distance from a point or spherical mass is given by:
	\[
		g = -\frac{GM}{r^2}
	\]
	\item The orbital period of a satellite can be found using the equation:
	\[
		T = \frac{2 \pi r^2}{GM}
	\]
\end{itemize}


\section*{Newton's law of gravitation}
\textbf{Definition:} Any two point masses attract each other with a force that is directly proportional to the product of their masses and inversely proportional to the square of their separation.

In my opinion that's a bit wordy, so my advice is to define the equations and state the terms. If the question wants lots of marks then state that it's a inverse square law too.

\end{document}
