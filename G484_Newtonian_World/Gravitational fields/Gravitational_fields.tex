\documentclass{article}

\usepackage[normalem]{ulem}
\usepackage{fancyhdr}
\usepackage[parfill]{parskip}
\usepackage{amsmath}
\pagestyle{fancyplain}

\title{Gravitational fields}
\author{Todd Davies}
\date{\today}

\begin{document}

\rhead{Gravitational fields}
\lhead{\today}

\maketitle

\section*{What is gravity?}
\thispagestyle{empty}
Gravity is the attraction between all matter and is one of the four fundamental
forces of the universe. Large objects such as planets have a very strong
gravity, while smaller objects have a weaker gravity. Gravity has an inverse
square relationship between strength and distance, the closer two objects get to
each other, the larger the force between them.

Gravity can be represented by field lines. These have arrows that show the
direction of the gravitational field on a mass placed in the field, and the
spacing of the lines indicates the strength of the field (a more dense spacing
corresponds to a stronger field).

We can treat most objects as a point mass when we're working with them in
relation to the gravitational fields they produce and interact with. The point
is centred around the centre of gravity of the object.


\section*{Equations you need to know}

\begin{itemize}
	\item The gravitational field strength at a point is equal the the
	gravitational force exerted per unit mass on an object placed at the point:
	\[
		g = \frac{F}{m}
	\]
	\item The equation for Newton's law of gravitation states:
	\[
		F = -\frac{GMm}{r^2}
	\]
	\textit{N.b. the minus sign is to show that the force is attractive.}
	\item The gravitational field strength at a specified distance from a point
	or spherical mass is given by:
	\[
		g = -\frac{GM}{r^2}
	\]
	\item Kepler's third law of planetary motion relates the orbital period to
	the orbital radius in the following equation:
	\[
		T^2 \propto r^3
	\]
	\item The orbital speed of a planet is determined using the equation:
	\[
		v^2 = \frac{GM}{r}
	\]
\end{itemize}

The quantities and their respective units are shown in this table:

\begin{center}
	\begin{tabular}{|l|l|l|}
		\hline
			Quantity & Abbreviation & Unit \\ \hline
			Gravity & $g$ & Meters per second squared ($ms^{-2}$) \\ \hline
			Force & $F$ & Newton ($N$) \\ \hline
			Mass & $m$ & Kilogram ($kg$) \\ \hline
			Gravitational constant & $G$ & $m^3kg^{-1}s^{-2}$ \\ \hline
			Velocity & $v$ & $ms^{-1}$\\ \hline
			Mass of the object & M & $kg$ \\
			creating the gravitational field & & \\ \hline
	\end{tabular}
\end{center}
\textit{N.b. $G = 6.67 \times 10^{-11}$}



\section*{Newton's law of gravitation}
\textbf{Definition:} Any two point masses attract each other with a force that
is directly proportional to the product of their masses and inversely
proportional to the square of their separation.

In my opinion that's a bit wordy, so my advice is to define the equations and
state the terms. If the question wants lots of marks then state that it's a
inverse square law too. The word 'point mass' is interchangeable with the word
particle (which probably doesn't mean anything unless you've done a maths
mechanics).

We can derive the equation of Newton's law of gravitation like so:

Start with $F \propto Mm$ and $F \propto \frac{1}{r^2}$ and combine them to
give:
\[
	F \propto \frac{Mm}{r^2}
\]
If we get rid of the proportional sign by adding in the gravitational constant
(G), we end up with Newton's law of gravitation:
\[
	F = -\frac{GMm}{r^2}
\]

We measure the distance $r$ between the masses from the centre of gravity of one
point to the centre of gravity of another. The two bodies attract each other
with equal and opposite forces (think Newton's third law).

\section*{Gravitational field strength}
The gravitational field strength of an object tells us how strong or weak the
gravity produced by the object is. The gravitational field strength of an object
is represented in equations by the letter $g$.

The gravitational field strength of an object is \textbf{defined as}:
\textit{The gravitational field strength at a point is the gravitational force
exerted per unit mass on a small object placed at that point.}

The unit of gravitational field strength is $Nkg^{-1}$ or $ms^{-2}$. These units
are interchangeable since one Newton is defined as:
\[
	N = kg \frac{m}{s^2}
\] 
If we divide this by a kilogram we get:
\[
	\frac{N}{kg} = \frac{m}{s^2}
\]
Therefore $Nkg^{-1}$ and $ms^{-2}$ are equal to each other.


\subsection*{The relationship between gravity and distance}
Gravity has an inverse squared relationship with distance. We can see this from
the equation: $g = -\frac{GM}{r^2}$. 

If this is the case then if I climb some stairs from the first floor to the
second floor, then (assuming the floors are the same distance apart) gravity
would have one quarter the effect on me once I was on the second floor compared
to when I was on the first floor. Yes?

Of course, this isn't true in fact we can work out the \% change in gravity
between the two floors. I'm going to assume that the floors are each $10m$ high.

First we need to work out the force of gravity on me when I'm standing on each
floor:

\begin{minipage}[t]{0.5\textwidth}
First floor:
\[
	G = 6.67 \times 10^{-11}
\]
\[
	M = 6 \times 10^{24}kg \textrm{ Mass of the Earth}
\]
\[
	r = 6400010m \textrm{ Radius of Earth + 10m}
\]
\[
	g = \frac{GM}{r^2}
\]
\[
	g = \frac{6.67 \times 10^{-11} \times 6 × 10^{24}}{6400010^2}
\]
\[
	g = 9.77047728ms^{-2}
\]
\end{minipage}
\begin{minipage}[t]{0.5\textwidth}
Second floor:
\[
	G = 6.67 \times 10^{-11}
\]
\[
	M = 6 \times 10^{24}kg \textrm{ Mass of the Earth}
\]
\[
	r = 6400020m \textrm{ Radius of Earth + 20m}
\]
\[
	g = \frac{GM}{r^2}
\]
\[
	g = \frac{6.67 \times 10^{-11} \times 6 \times 10^{24}}{6400020^2}
\]
\[
	g = 9.77044675ms^{-2}
\]
\end{minipage}\\

From these two values, we can find the percentage decrease in gravity between
the two floors:
\[
	\textrm{Percentage decrease} 
	= \frac{9.77044675 - 9.77047728}{9.77047728} \times 100
\]
\[
	\textrm{Percentage decrease} = 0.0003 \%
\]
So we've proven that even though gravity has an inverse square relationship with
distance, because the distance you move by going up by (ten metres) is very
small compared to the radius of the Earth, the pull of gravity decreases by only
a fraction of a percent.

\section*{Orbiting under gravity}
Gravity provides the centripetal force required to keep celestial bodies moving
around each other. We can use equations relating to gravity along with equations
relating to circular motion to work with orbits of satellites around planetary
bodies. Since gravity provides the centripetal force in these equations, we can
combine $F = \frac{mv^{2}}{r}$ with $F = \frac{GMm}{r^2}$ to get:
\[
	\frac{GMm}{r^2} = \frac{mv^{2}}{r}
\]
Which we can rearrange to give:
\[
	v^{2} = \frac{GM}{r}
\]
Now we can calculate things such as the required speed of a satellite in orbit,
or the mass of the object a satellite is orbiting around.

\section*{Orbital period}
Kepler's third law of planetary motion states: \textit{The square of the period
T of a planet is directly proportional to the cube of it's distance r from the
Sun - $T^2 \propto r^3$}

The constant in Kepler's law is $\frac{4 \pi ^2}{GM}$ which is also (due to the
nature of a straight line graph) the gradient of the line we get when we plot a
graph of $T^2$ against $r^3$.

\section*{Orbits of satellites}
Satellites can orbit the earth in any orbit, but some are more useful than the
others.

If a satellite is set to orbit over the poles, then it passes around the Earth
many times a day. By doing this, the satellite views a slightly different strip
of earth every orbit as the Earth is rotating underneath it.

A satellite can be in an elliptical orbit around Earth. This has a longer time
period than a circular orbit since the satellite must travel further and farther
away from the planet in every orbit.

Arguably the most useful orbit is the geostationary orbit. By matching the time
period of the satellite to the speed of the Earth's rotation, we can make a
satellite orbit over the same spot on the Earth.

\section*{Deriving Kepler's law}
\textbf{This isn't on the syllabus, but if you're extra clever you might like to
read it}

Kepler's law states that: $T^2 \propto r^3$. Here is how to derive it from the
following three equations:

\begin{itemize}

\item Newton's law of gravitation: $F = -\frac{GMm}{r^2}$

\item The centripetal force equation: $F = \frac{mv^2}{r}$

\item The equation for the speed of an object travelling in a circle: 
$v = \frac{2 \pi r}{T}$

\end{itemize}


Put Newton's law of gravitation and the centripetal force equation equal to each
other:

$\frac{GMm}{r^2} = \frac{mv^2}{r}$

Divide both sides by $r$:

$\frac{GMm}{r} = mv^2$

Sub in $v = \frac{2 \pi r}{T}$ for $v$:

$\frac{GMm}{r} = m(\frac{2 \pi r}{T})^2$

Divide both sides by $m$:

$\frac{GM}{r} = (\frac{2 \pi r}{T})^2$

Root both sides:

$\sqrt{\frac{GM}{r}} = \frac{2 \pi r}{T}$

Flip both sides and multiply by $2 \pi r$:

$T = \frac{2 \pi r}{\sqrt{\frac{GM}{r}}}$

Now we have Kepler's law, sort of, we it just needs simplifying:

Multiply both sides by $\sqrt{\frac{GM}{r}}$:

$T \times \sqrt{\frac{GM}{r}} = 2 \pi r$

Square both sides:

$T^2 \times \frac{GM}{r} = (2 \pi r)^2$

Divide both sides by $\frac{GM}{r}$:

$T^2 = \frac{(2 \pi r)^2}{\frac{GM}{r}}$

Clean it up:

$T^2 = \frac{(2 \pi r)^2 \times r}{GM}$

Take out $r$ to get the final answer:

$T^2 = \frac{(2 \pi)^2}{GM}r^3$

If you take out the constant you get Kepler's law:

$T^2 \propto r^3$

\section*{Questions to practice}
\textbf{Question 1}\\
Work out the radius of orbit of a geostationary satellite.
($G = 6.67 \times 10^{-11}$, $M = 6 \times 10^{24}$).

\textbf{Answer}\\
Kepler's law states that:
\[
	T^2 = \frac{4 \pi ^2}{GM} r^3
\]
So if we assume a geostationary orbit has a period of 24 hours and sub in the
numbers we know:
\[
	(86400)^2 = \frac{
					4 \pi ^2
				}{
					6.67 \times 10^{-11} \times 6 \times 10^{24}
				} r^3
\]
Rearrange to find r:
\[
	r = \sqrt[3]{
			\frac{
				(86400)^2
			}{
				\frac{
					4 \pi ^2
				}{
					6.67 \times 10^{-11} \times 6 \times 10^{24}
				}
			}
		}
\]
\[
	r = \sqrt[3]{
			\frac{
				(86400)^2 \times 6.67 \times 10^{-11} \times 6 \times 10^{24}
			}{
				4 \pi ^2
			}
		}
\]
\[
	r = 4.23 \times 10^{7}m
\]
\textbf{Question 2}\\
Work out the latency of a internet connection between Paris and London
assuming one Geostationary satellite is used to facilitate the connection and
there is no processing delay. (Speed of light = 
$3 \times 10^8 ms$, $G = 6.67 \times 10^{-11}$, $M = 6 \times 10^{24}$)

\textbf{Answer}\\
We already worked out the radius of an orbiting satellite in the previous
question to be $4.23 \times 10^{7}m$ We just need to work out the time it takes
for light to cover this distance and back again (the signal must go up and
down).
\[
	speed = \frac{distance}{time}
\]
\[
	3 \times 10^8 = \frac{4.23 \times 10^{7}}{time}
\]
\[
	time = 0.141s
\]
\textbf{Question 3}\\
Calculate the two distances where the gravitational field strength of the Earth
is $1N/kg$. ($G = 6.67 \times 10^{-11}$, $M = 6 \times 10^{24}$)

\textbf{Answer}\\
We can use the equation:
\[
	g = -\frac{GM}{r^2}
\]
Sub in the numbers we know:
\[
	1 = -\frac{(6.67 \times 10^{-11}) \times (6 \times 10^{24})}{r^2}
\]
\[
	r = \sqrt{\frac{(6.67 \times 10^{-11}) \times (6 \times 10^{24})}{1}}
\]
\[
	r = 20004999.4m
\]
\[
	r = 20,005km
\]
However, this is the distance away from the Earth that $g = 1N/kg$. We also need
to find the distance into the Earth for this value of $g$:
Since $g = 9.8N/kg$ when $r = 6400km$ and the relationship is directly
proportional we can work out the gravitational field strength of any point
inside the Earth.
For $g = 1N/kg$ we divide $9.8$ by $1$ to get $9.8$ and then divide $6400km$ by
this:
\[
	r = \frac{6400000}{9.8} = 653061m = 653km
\]
\end{document}