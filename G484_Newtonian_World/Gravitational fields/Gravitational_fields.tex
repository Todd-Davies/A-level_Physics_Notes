\documentclass{article}

\usepackage[normalem]{ulem}
\usepackage{fancyhdr}
\usepackage[parfill]{parskip}
\pagestyle{fancyplain}

\title{Gravitational fields}
\author{Todd Davies}
\date{\today}

\begin{document}

\rhead{Circular motion}
\lhead{\today}
\setlength{\parindent}{0cm}

\maketitle

\section*{What is gravity?}
\thispagestyle{empty}
Gravity is the attraction between all matter and is one of the four fundamental forces of the universe. Large objects such as planets have a very strong gravity, while smaller objects have a weaker gravity. Gravity has an inverse square relationship between strength and distance, the closer two objects get to each other, the larger the force between them.

Gravity can be represented by field lines. These have arrows that show the direction of the gravitational field on a mass placed in the field, and the spacing of the lines indicates the strength of the field (a more dense spacing corresponds to a stronger field).

We can treat most objects as a point mass when we're working with them in relation to the gravitational fields they produce and interact with. The point is centred around the centre of gravity of the object.



\end{document}
