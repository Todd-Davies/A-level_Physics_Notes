\documentclass{article}

\usepackage[normalem]{ulem}
\usepackage{fancyhdr}
\usepackage{multirow}
\usepackage[parfill]{parskip}
\pagestyle{fancyplain}

\title{Ideal gases}
\author{Todd Davies}
\date{\today}

\begin{document}

\rhead{Ideal gases}
\lhead{\today}

\maketitle

\section*{What is an ideal gas?}
\thispagestyle{empty}
An ideal gas is a model of a gas we use that behaves according to the equations
$pV = nRT$ and $pV = NkT$.

\section*{Equations to remember}
\begin{itemize}
	\item To convert between kelvin and celcius, use the following:
	\[
		T(K) = \Theta(^\circ C) + 273
	\]
	\item For an ideal gas:
	\[
		\frac{pV}{T} = constant
	\]
	\item The ideal gas equation is:
	\[
		pV = NkT
	\]
	or
	\[
		pV = nRT
	\]
	\item You can find the mean translational kinetic energy of a gas by using:
	\[
		E = \frac{3}{2}kT
	\]
	\end{itemize}

The quantities and their respective units are shown in this table:

\begin{center}
	\begin{tabular}{|l|l|l|}
		\hline
			Quantity & Abbreviation & Unit \\ \hline
			Absolute temperature & $K$ & Kelvin ($K$) \\ \hline
			Temperature & $^\circ C$ & Celsius ($^\circ C$) \\ \hline
			Pressure & $p$ & Pascals ($Pa$) / $Nm^{-2}$ \\ \hline
			Volume & $V$ & Meters cubed($m^3$)\\ \hline
			Number of particles & $N$ & $N/A$\\ \hline
			Boltzmann's constant & $k$ & Joules per kelvin ($JK^{-1}$)\\ \hline
			Number of moles & $n$ & Moles ($Mol$)\\ \hline
			Energy & $E$ & Joules ($J$)\\ \hline
	\end{tabular}
\end{center}

\section*{Measuring gasses}
A gas has many quantities, such as temperature, pressure, density etc. What are
they and how do we quantify them?

\begin{itemize}
	\item \textbf{Pressure} is the force exerted by the gas on the walls of it's
	container (we know about this from Thermal Physics). The units of pressure
	are Pascals ($Pa$) or $Nm^{-2}$.
	\item \textbf{Temperature} is the amount of heat energy the gas has, or the
	average amount of kinetic energy each molecule in the gas has. Remember to
	work in Kelvin for all the equations in this topic! The equation to convert
	between Celsius and Kelvin is: 
	\[
		T(K) = \Theta(^\circ C) + 273
	\]
	\item \textbf{Volume} it the amount of space the gas takes up. Measured in
	$m^3$.
	\item \textbf{Mass} The mass of the gas is measured in $kg$ or $g$.
	Alternately, we can measure the amount of gas, the unit of which is $moles$.
	
	The definition of a mole is: \textit{One mole of any substance is the amount
	of that substance which contains the same number of particles as there are
	in 0.012kg of carbon-12.}
	
	This also means that one mole of a substance weighs it's atomic mass in
	grams (e.g. one mole of oxygen gas weighs $32g$). One mole is also known as
	Avogadro's constant and is equal to $6.01 \times 10^{23}$
\end{itemize}

\section*{Boyle's law}
Boyle's law is defined as: \textit{The pressure exerted by a fixed mass of gas
is inversely proportional to its volume, provided the temperature of the gas
remains constant.}

It can also be written as $p \propto \frac{1}{V}$ or $pV = constant$.

This isn't very useful on it's own, but you can derive this equation:
\[
	p_1V_1 = p_2V_2
\]

Where the initial pressure and volume are denoted by the subscript \textit{1}
and the final pressure and volume are denoted by the subsctipt \textit{2}.

\section*{Charles' law}
Boyle's law requires the temperature of the gas to be constant, so what about if
the temperature changes? Charles' law is an equation to take this into account:
\[
	V \propto T
\]
\[
	\frac{V}{T} = constant
\]
However, this relationship only holds when the amount of gas is fixed and the
pressure is constant.

\section*{Combining Boyle's law and Charles' law}
We can combine these two equations to get:
\[
	\frac{pV}{T} = constant
\]
Or using the second version of Boyle's law:
\[
	\frac{p_1V_1}{T_1} = \frac{p_2V_2}{T_2}
\]

\section*{Assumptions about ideal gasses}
There are some assumptions we must make about ideal gasses for these equations
to hold. Unfortunately, examiners like to make you recite these:

\begin{itemize}
	\item A gas contains a very large number of spherical particles.
	\item The forces between particles are negligible, except during collisions.
	\item The volume of the particles is negligible compared to the volume
	occupied by the gas.
	\item Most of the time, a particle moves in a straight line at a constant
	velocity. The time of collision with each other or the container walls is
	negligible compared with the time between collisions.
	\item Collisions of particles with each other and with the container are
	perfectly elastic, so no kinetic energy is lost.
\end{itemize}

\section*{The ideal gas equation}
We have an equation that takes the amount of gas we have into account as well
as the temperature, volume and pressure. This is the ideal gas equation. There
are actually two different forms of the equation we can use.

For particles, we use this one:
\[
	pV = NkT
\]	
Where $k$ is Boltzmann's constant which is equal to
$1.38 \times 10^{-23}JK^{-1}$

When we're working in moles, we can use:
\[
	pV = nRT
\]
Where $n$ is the number of moles and $R$ is the universal molar gas constant
which is equal to $8.31Jmol^{-1}K^{-1}$.

The Boltzmann constant, Avogadro constant and the universal molar gas constant
are related using:
\[
	k = \frac{R}{N_A}
\]

\subsection*{Finding the number of moles}
To use the ideal gas equation, we need the number of moles present in the gas.
Often we need to work it out and can do so using this equation:
\[
	\textrm{\textit{Number of moles}} = \frac{mass(g)}{\textrm{\textit{molar
	mass(g $mol^{-1}$)}}}
\]

\section*{Temperature and the kinetic energy of molecules}
We now know about and can work with the macroscopic quantities of gases, but
what about them on a microscopic level?

There is a relationship between the temperature of a gas and the kinetic energy
of it's particles:

\textit{The mean translational kinetic energy of an atom or molecule in a ideal
gas is proportional to the thermodynamic temperature of the gas.}

\textit{N.b. Translational kinetic energy is velocity, don't confuse it with
rotational kinetic energy which is angular velocity.}

We can also write that as:
\[
	\textrm{\textit{Mean translational kinetic energy of particle}} \propto T
\]
If we put in a constant, we get this:
\[
	E = \frac{3}{2}kT
\]

\section*{Questions to practice}
\textbf{Question 1}\\
Air consists of Oxygen (Molecular mass = 32$amu$) and Nitrogen (Molecular mass =
28$amu$). Calculate the two mean translational kinetic energies of Oxygen and
Nitrogen at 20($^\circ C$)

\textbf{Answer}\\
Use $E = \frac{3}{2}kT$\\
Energy = $\frac{3}{2} \times (1.38 \times 10^{-23}) \times (20+273) = 6.07
\times 10^{-21}$

Use $KE = \frac{1}{2}mv^2$:\\
Oxygen: 
\[
	\sqrt{
		\frac{
			2 \times (6.07 \times 10^{-21})
		}{
			(0.032 \div 6.02\times 10^{23})
		}
	} = 480m/s
\]
Nitrogen: 
\[
	\sqrt{
		\frac{
			2 \times (6.07 \times 10^{-21})
		}{
			(0.028 \div 6.02\times 10^{23})
		}
	} = 511m/s
\]

\textbf{Question 2}\\
A fixed mass of gas expands to three times it's original volume at a constant
temperature. How does the pressure and average kinetic energy of the particles
change?

\textbf{Answer}\\
Pressure is divided by three. Since $pV = constant$.

The average kinetic energy is constant since it is dependant on temperature.

\textbf{Question 3}\\
What happens to the speed of the particles in a gas if the temperature is
quadrupled?

\textbf{Answer}\\
We can combine $E = \frac{3}{2}kT$ with $E = \frac{1}{2}mv^2$:
\[
	\frac{3}{2}kT = \frac{1}{2}mv^2
\]
Lets take out all the constants:
\[
	T \propto v^2
\]
Multiply both sides by four (to quadruple the temperature)
\[
	4T = 4v^2
\]
Find how much $v$ has increased by:
\[
	4 = v^2
\]
\[
		v = 2
\]
Therefore the velocity has doubled.
\end{document}